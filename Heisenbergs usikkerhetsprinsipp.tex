\documentclass{article}

% Language setting
\usepackage[norwegian]{babel}

\usepackage[a4paper,top=2cm,bottom=2cm,left=3cm,right=3cm]{geometry}

% Useful packages
\usepackage{amsmath, amssymb}
\usepackage{minted}
\usepackage{graphicx}
\usepackage{xcolor}
\usepackage{paracol} 
\usepackage{siunitx}
\usepackage[colorlinks=true, allcolors=blue]{hyperref}
\usepackage{gensymb}
\usepackage{setspace}
\usepackage{float}
\usepackage{parskip}

\newcommand{\inner}[1]{\ensuremath{\left\langle #1 \right\rangle}}


\begin{document}
       \begin{titlepage}
		\centering
		\vspace*{1in}

		{\Huge \textbf{Utledning av Heisenbergs Usikkerhetsprinsipp}} \\
		\vspace{0.5in}


		{\Large Hannah Angela Ekblad} \\
		\vspace{0.5in}
        {\Large 17. mars 2025} \\
		\vspace{0.4in}
    

		\vfill
		\begin{center}
		    \includegraphics[width=0.69\textwidth]{meme.png}
		\end{center}
		
		\vspace{1in}
    \end{titlepage}


\tableofcontents
\newpage

\section{Innledning}

Heisenbergs usikkerhetsprinsipp postulerer at posisjon og bevegelsesmengde ikke kan måles nøyaktig samtidig. Matematisk kan dette uttrykkes som:

\begin{equation*}
    \Delta x \Delta p \geq \frac{\hbar}{2}
\end{equation*}

der $\Delta x$ er usikkerheten i posisjon, $\Delta p$ er usikkerheten i bevegelsesmengde, og $\hbar$ er den reduserte Planck-konstanten. Dette prinsippet er svært viktig innen kvantemekanikken, da det setter en grense for hvor presist man kan kjenne tilstanden til et system. 

Rapporten er todelt. Første del fokuserer på utledningen av Heisenbergs usikkerhetsprinsipp, mens andre del demonstrerer hvordan prinsippet fungerer ved hjelp av et enkelspalteforsøk. Gjennom dette eksperimentet kan man se hvordan bølgefunksjonens utstrekning i posisjonsrommet påvirker fordelingen i bevegelsesmengderommet, og dermed gi en forståelse av usikkerhetsrelasjonen.

\vspace{0.2in}



\section{Utledning av Heisenbers usikkerhetsprinsipp}
I denne delen av rapporten skal jeg vise hvordan Heisenbergs usikkerhetsprinsipp oppstår matematisk ved hjelp av Fouriertransformasjon og Plancherels teorem. Denne utledningen bygger på Cauchy-Schwarz’ ulikhet og viser at dersom en bølgefunksjon er sterkt lokalisert i posisjonsrommet, vil den være bredt spredt i bevegelsesmengderommet, og vice versa. Dvs. jo mer presist vi måler en partikkels posisjon, desto mer usikker blir dens bevegelsesmengde.  

\vspace{0.2 in}

\subsubsection{Fouriertransformasjonen og dens invers}
For å analysere sammenhengen mellom posisjon og bevegelsesmengde, uttrykker vi en bølgefunksjon $\Psi(x)$ i posisjonsrommet som en Fouriertransformasjon
\begin{equation}
    \widehat{\Psi}(p_x) = (2\pi\hbar)^{-1/2}\int_{-\infty}^{\infty}\Psi(x)e^{-ip_xx/\hbar}\, dx \label{eq:1}
\end{equation}
Her representerer $\widehat{\Psi}(p_x)$ bølgefunksjonen i bevegelsesmengderommet. Invers Fouriertransformasjonen gir oss tilbake funksjonen i posisjonsrommet
\begin{equation}
    \Psi(x) = (2\pi\hbar)^{-1/2}\int_{-\infty}^{\infty} \widehat{\Psi}(p_x)e^{ip_xx/\hbar}\, dp_x \label{eq:2}
\end{equation}


\subsubsection{Plancherels teorem}
Videre trenger vi Plancherels teorem. Denne sier at integralet av kvadratet av en funksjon er det samme i både posisjons- og bevegelsesmengderommet:

\vspace{0.1in}

\begin{equation}
    \int_{-\infty}^{\infty}|\widehat{\Psi}(p_x)|^2 \, dp_x = \int_{-\infty}^{\infty}|\Psi(x)|^2\, dx \label{eq:3}
\end{equation}

\vspace{2.3in}

Dette kan vises ved utregning:

Vi starter med å skrive ut normen og benytter Fourier-inversen
\[
\begin{aligned}
    \int_{- \infty}^{\infty}| \hat{\Psi}(p_x)|^2 dp_x &= \frac{1}{2\pi\hbar} \int_{- \infty}^{\infty} |\Psi(x)| e^{-ip_x x/\hbar} dx
     \cdot \int_{- \infty}^{\infty} \overline{\Psi}(s)e^{ip_x s/\hbar} ds \, dp_x \\
    &= \frac{1}{2\pi\hbar} \int_{- \infty}^{\infty}\int_{- \infty}^{\infty} \Psi(x) \overline{\Psi}(s) \int e^{ip_x(s-x)/\hbar} dp_x \, dx \, ds \\
    &=\frac{1}{2\pi\hbar} \int_{- \infty}^{\infty} \Psi(x) \overline{\Psi}(s) \, (2\pi\hbar\delta(s-x)) dp_x \, dx \, ds \\
    &= \int_{- \infty}^{\infty}\int_{- \infty}^{\infty} \Psi(x) \overline{\Psi}(s) \, \delta(s-x) \, dx \, ds\\
    &= \int_{- \infty}^{\infty}\int_{- \infty}^{\infty} \Psi(x) \overline{\Psi}(x) \, dx \\
    &= \int_{-\infty}^{\infty} |\Psi(x)|^2 dx.
\end{aligned}
\]

Dirac-deltafunksjonen gjør at integralet kollapser i linje 2 av utledningen. 
Plancherels teorem viser dermed at normbevaringen opprettholdes mellom de to representasjonene av bølgfunksjonen. 

\vspace{0.2in}

\subsubsection{Derivasjon av Fourertransformasjonen}
For å gå videre trenger vi derivasjonsregelen av Fouriertransformasjonen:
\begin{equation}
    \widehat{\Psi'}(p_x) = \frac{ip_x}{\hbar}\widehat{\Psi}(p_x) \label{eq:4}
\end{equation}

Denne regelen utledes slik:
\[
\begin{aligned}
    \widehat{\Psi'} (p_x) &= \mathfrak{F}\{\Psi'(x)\} \\
    &= (2\pi\hbar)^{-1/2} \int_{-\infty}^{\infty} \Psi'(x)e^{-ip_xx/\hbar} \, dx \\
    &= (2\pi\hbar)^{-1/2} \Big[ \Psi(x)e^{-ip_xx/\hbar} \Big]_{x=-\infty}^{x=\infty} 
    + \frac{ip_x}{\hbar} (2\pi\hbar)^{-1/2} \int_{-\infty}^{\infty} \Psi(x)e^{-ip_xx/\hbar} \, dx \\
    &= \frac{ip_x}{\hbar} \widehat{\Psi}(p_x).
\end{aligned}
\]

I linje 2 - 3, gjør vi en delvis integrasjon, og pga antakelsen vi gjør om at $\lim_{x \rightarrow \pm \infty} |\Psi(x)| = 0$, vil første ledd i linje 3 forsvinne. Da står vi kun igjen med siste ledd, som vi ser er Fouriertransformasjonen $\widehat{\Psi}(p_x)$ fra likning \ref{eq:1}.

\vspace{0.2in}

\subsubsection{Viktig formel}
Vi trenger også denne med oss videre
\begin{equation}
    \left| \Psi(x) \right|^2 = \Psi(x)\overline{\Psi}(x) \label{eq:5}
\end{equation}

\vspace{0.2in}

\subsubsection{Standardavvik for posisjon og bevegelsesmengde}
Standardavviket til posisjon er definert som
\begin{equation*}
    \Delta x = \sqrt{\langle x^2 \rangle - \langle x \rangle^2}
\end{equation*}
For enkelhets skyld antar vi at bølgefunksjonen er sentrert rundt null, altså $\inner{x} = 0$, slik at

\begin{align*}
     \Delta x &= \sqrt{\langle x^2 \rangle} \\
     &= \sqrt{\langle \Psi | x^2 \Psi \rangle} \\
     &= \sqrt{\int_{-\infty}^{\infty} \overline{\Psi}x^2\Psi \, dx} \\
     &= \sqrt{\int_{-\infty}^{\infty}x^2\Psi \overline{\Psi} \, dx} \\
     &= \sqrt{\int_{-\infty}^{\infty}x^2 |\Psi|^2\, dx}
\end{align*}

Dermed blir standardavviket (usikkeherheten) til posisjonen uttrykt som:
\begin{equation} 
    \Delta x = \sqrt{\int_{-\infty}^{\infty}x^2|\Psi(x)|^2\, dx} \label{eq:6}
\end{equation}
\vspace{0.2in}

Tilsvarende kan vi regne ut standardavviket til bevegelsesmengden fra
\begin{align*}
    \Delta p_x &= \sqrt{\langle p_x^2\rangle - \langle p_x\rangle^2} 
\end{align*}
hvor vi, på samme måte som for $\inner{x}$, antar at $\inner{p_x} = 0$. 
\begin{align*}
    \Delta p_x &= \sqrt{\langle p_x^2 \rangle} \\
    &= \sqrt{\int_{-\infty}^{\infty}\overline{\Psi}p_x^2\Psi \, dp_x} \\
    &= \sqrt{\int_{-\infty}^{\infty}p_x^2\Psi \overline{\Psi} \, dp_x} \\
    &= \sqrt{\int_{-\infty}^{\infty}p_x^2 |\Psi|^2\, dp_x} \\
    &= \sqrt{\int_{-\infty}^{\infty}p_x^2 | \Psi(x) |^2 \, dp_x}
\end{align*}
Dermed blir standardavviket (usikkeherheten) til bevegelsesmengden uttrykt som:
\begin{equation}
    \Delta p_x = \sqrt{\int_{-\infty}^{\infty}p_x^2|\widehat{\Psi}(p_x)|^2\, dp_x} \label{eq:7}
\end{equation}

I begge utledningene over, brukte jeg den viktige formelen (\ref{eq:5}) for å kunne skrive at $\Psi\overline{\Psi} = |\Psi|^2$.

\vspace{0.2in}
\subsubsection{Andre relasjoner}
Vi trenger også følgende relasjoner for å kunne anvende Cauchy-Schwarz' ulikhet

\begin{equation}
    |(f,g)|^2 \le (f,f) (g,g), \quad |a+b| \le |a| + |b|, \quad \left|\int_a^bf(x)\,dx\right| \le \int_a^b|f(x)|\,dx \label{eq:8}
\end{equation}



\subsubsection{Selve utledningen}

\begin{align}  
    1 &= |1|  \label{eq:start} \\
    &= \left| \int_{-\infty}^{\infty} |\Psi(x)|^2 dx \right| \label{eq:11} \\
    &= \left| \int_{-\infty}^{\infty} |\Psi(x)|^2 \frac{d}{dx}x \, dx \right| \label{eq:12} \\
    &= \left| \underbrace{\big [ x \, |\Psi(x)|^2 \big]_{x = -\infty}^{x = \infty}}_{= 0} -\int_{-\infty}^{\infty}x \frac{d}{dx}|\Psi(x)|^2 \, dx\right| \label{eq:13}\\
    &= \left| -\int_{-\infty}^{\infty}x \frac{d}{dx}|\Psi(x)|^2 \, dx\right| \label{eq:14}\\
    &= \left| \int_{-\infty}^{\infty} x \frac{d}{dx} |\Psi(x)|^2 dx \right| \label{eq:15}\\
    &= \left| \int_{-\infty}^{\infty} x \frac{d}{dx}\Big( \Psi(x) \overline{\Psi}(x) \Big) \, dx \right| \label{eq:16}\\
    &= \left| \int_{-\infty}^{\infty}x \Big(\Psi'(x)\overline{\Psi}(x) + \Psi(x) \overline{\Psi'}(x) \Big) \, dx \right| \label{eq:17}\\
    &= \left| \int_{-\infty}^{\infty} x \Psi'(x) \overline{\Psi}(x) + x \Psi(x) \overline{\Psi'}(x) \, dx \right| \label{eq:18} \\
    &\leq \left|\int_{-\infty}^{\infty}x \Psi'(x) \overline{\Psi}(x) \, dx \right| + \left| \int_{-\infty}^{\infty}x \Psi(x) \overline{\Psi'}(x) \, dx\right| \label{eq:19}\\
    &\leq \int_{-\infty}^{\infty}\left|x \Psi'(x) \overline{\Psi}(x)  \right| \, dx + \int_{-\infty}^{\infty}\left|x \Psi(x) \overline{\Psi'}(x)\right|\, dx \label{eq:20}\\
    &\leq \int_{-\infty}^{\infty}\left|x \right| \left|\Psi'(x)\right| \left\overline{\Psi}(x)  \right| \, dx + \int_{-\infty}^{\infty}\left|x \right| \left|\Psi(x) \right| \left| \overline{\Psi'}(x)\right|\, dx \label{eq:21}\\
    &\leq \int_{-\infty}^{\infty}\left|x \right| \left|\Psi'(x)\right| \left \Psi(x)  \right| \, dx + \int_{-\infty}^{\infty}\left|x \right| \left|\Psi(x) \right| \left| \Psi'(x)\right|\, dx \label{eq:22}\\
    &\leq \int_{-\infty}^{\infty}\left|x \right| \left| \Psi(x)  \right| \left|\Psi'(x)\right|\, dx + \int_{-\infty}^{\infty}\left|x \right| \left|\Psi(x) \right| \left| \Psi'(x)\right|\, dx \label{eq:23}\\
    &\leq 2 \int_{-\infty}^{\infty} |x| |\Psi(x)| |\Psi'(x)| dx \label{eq:24}\\
    &\leq 2 \sqrt{\int_{-\infty}^{\infty} x^2 |\Psi(x)|^2 dx} \sqrt{\int_{-\infty}^{\infty} |\Psi'(x)|^2 dx} \label{eq:25}\\
    &= 2 \sqrt{\int_{-\infty}^{\infty} x^2 |\Psi(x)|^2 dx} \sqrt{\int_{-\infty}^{\infty} |\widehat{\Psi'}(p_x)|^2 dp_x} \label{eq:26}\\
    &= 2 \sqrt{\int_{-\infty}^{\infty} x^2 |\Psi(x)|^2 dx} \sqrt{\int_{-\infty}^{\infty} \left|\frac{ip_x}{\hbar} \widehat{\Psi}(p_x)\right|^2 dp_x} \label{eq:27}\\
    &= 2 \sqrt{\int_{-\infty}^{\infty} x^2 |\Psi(x)|^2 dx} \sqrt{\int_{-\infty}^{\infty} \frac{p_x^2}{\hbar^2} |\widehat{\Psi}(p_x)|^2 dp_x} \label{eq:28}\\
    &= \frac{2}{\hbar} \sqrt{\int_{-\infty}^{\infty} x^2 |\Psi(x)|^2 dx} \sqrt{\int_{-\infty}^{\infty} p_x^2 |\widehat{\Psi}(p_x)|^2 dp_x} \,\,\,\,\,\,\,\,  \cdot\frac{\hbar}{2} \label{eq:29}\\
    &\Rightarrow \sqrt{\int_{-\infty}^{\infty} x^2 |\Psi(x)|^2 dx} \sqrt{\int_{-\infty}^{\infty} p_x^2 |\widehat{\Psi}(p_x)|^2 dp_x} \geq \frac{\hbar}{2} \label{eq:30}\\
    &\Rightarrow \Delta x \Delta p \geq \frac{\hbar}{2} \label{eq:31} 
\end{align}




\subsubsection{Forklaring av stegene i selve utledningen}
En kortfattet forklaring av stegene over: I likning (\ref{eq:11}) benytter jeg normaliseringsbetingelsen for bølgefunksjonen. I likning (\ref{eq:12}) satt jeg inn den deriverte av x fordi det går an, og fordi jeg er en fri mann i et fritt land :). I likning (\ref{eq:13}) bruker jeg delvis integrasjon, med $x$ som $u'$ og $\Psi(x)$ som $v$ i henhold til formelen $\int u' v = uv - \int uv'$. Det første leddet blir null fordi vi krever at $\lim_{x\to\pm\infty} \Psi(x)=0$. Likning (\ref{eq:14}) og (\ref{eq:15}) rydder opp i utrykket, og minustegnet forsvinner på grunn av absoluttverien. I likning (\ref{eq:16}) brukte jeg formelen fra likning (\ref{eq:5}). I likning (\ref{eq:17}) benytter jeg produktregelen, og i likning (\ref{eq:18}), ganget jeg inn $x$. I likning (\ref{eq:19}) og (\ref{eq:20}) brukte jeg trekantulikheten for integraler for å sette aboluttverdier på plass. I likning (\ref{eq:21}), (\ref{eq:22}) og (\ref{eq:23}) brukte jeg formelen $|ab| = |a||b|$, og fjernet komplekskonjugasjonen, som ikke påvirker absoluttverdien. I likning (\ref{eq:24}) kombineres de to integralene siden de er like. I likning (\ref{eq:25}) anvender jeg Cauchy-Schwarz' ulikhet på integralet. I likningene (\ref{eq:26}) og (\ref{eq:27}) bruker jeg derivasjonsregelen for Fouriertransformasjonen. I likning (\ref{eq:28}) identifiserer vi uttrykkene for usikkerheten i posisjon $\Delta x$ og usikkerheten i bevegelsesmengde $\Delta p$. I likningene (\ref{eq:29}), (\ref{eq:30}) og (\ref{eq:31}), gjennkjenner vi Heisenbergs usikkerhetsprinsipp. 




\vspace{0.69in}

\section{Demonstrasjon av Heisenbergs usikkerhetsprinsipp ved et enkelspalteforsøk }

\vspace{0.1in}

\subsection{Henisikt}
Hensikten ble godt oppsummert i underoverskriften over. Ved hjelp av et enkelspalteforsøk skal jeg demonstrere Heisenbergs usikkerhetsprinsipp. Jeg undersøker hvordan endringer i spalteåpningen påvirker eksperimentet, og viser hvordan usikkerheten i posisjon og bevegelsesmengde henger sammen.

\vspace{0.2in}

\subsection{Teori}
For å forstå Heisenbergs usikkerhetsprinsipp, er det greit å vite litt om mannen bak. Werner Heisenberg (1901-1976) var en tysk fysiker, som i 1927, formulerte usikkerhetsprinsippet som beskriver en begrensning i hvor presist vi kan måle posisjon og bevegelsesmengde.

Som beskrevet tidligere, sier Heisenbergs usikkerhetsprinsipp at det er umulig å måle en partikkels posisjon og bevegelsesmengde samtidig med uendelig presisjon. Matematisk kan dette uttrykkes ved relasjonen:

\begin{equation*}
    \Delta x \Delta p_x \geq \frac{\hbar}{2}.
\end{equation*}

Dette betyr at dersom vi prøver å redusere usikkerheten i posisjonen $\Delta x$, vil usikkerheten i bevegelsesmengden $\Delta p_x$ øke, og omvendt. Med andre ord er disse størrelsene omvendt proporsjonale. 

For å se dette tydeligere, kan vi dele ligningen på $\Delta p_x$:

\begin{equation*}
    \Delta x \geq \frac{\hbar}{2\Delta p_x}.
\end{equation*}

Her ser vi klart at dersom $\Delta p_x$ øker, må $\Delta x$ minke, og vice versa. Dette er kjernen i usikkerhetsprinsippet og er det jeg skal demonstrere eksperimentelt.  

Ved å bruke et enkelspalteforsøk kan vi vise at en reduksjon i spaltebredden (som tilsvarer en redusert usikkerhet i posisjon) fører til en større spredning av interferensmønsteret (som tilsvarer en økt usikkerhet i bevegelsesmengde). 

Et interferensmønster er resultatet av når bølger fra to eller flere kilder kombineres, hvor bølger i samme fase forsterker hverandre og danner en konstruktiv overlagring, og bølger i motsatt fase kansellerer hverandre gjennom en destruktiv overlagring. 

\begin{center}
    \includegraphics[width=0.35\textwidth]{ny_ny_interferens.png}\\
    \caption{\label{fig:Interferens} En forsøkt illustrasjon på et interferensmønster \\med konstruktiv og destruktiv overlagring.}
\end{center}
Toppunktene på intensitetsgrafen til høyre i bildet over tilsvarer lyse prikker eller striper på veggen, mens bunnpunktene tilsvarer mørke prikker eller striper.


\vspace{0.2in}
\subsection{Utstyr}
Siden jeg dessverre ikke har direkte tilgang (eller klassifisering) til å kjøpe en klasse II-laser, måtte jeg bruke fantasien og dykke ned i spesifikasjonene fra \href{https://www.fybikon.no/om-oss}{FYBIKON – din realfagsleverandør} – for å finne realistiske verdier for en laser og optiske gitre.


\vspace{0.1in}
Utstyr inkluderer dermed:
\begin{itemize}
\item To optiske gittre, en med \href{https://www.fybikon.no/fysikk/optikk/diffraksjon/optisk-gitter-300-linjer/mm}{300 linjer/mm} og en med \href{https://www.fybikon.no/fysikk/optikk/diffraksjon/optisk-gitter-600-linjer/mm}{600 linjer/mm}
\item En \href{https://www.fybikon.no/fysikk/optikk/laser/laserpenn-rod-klasse-ii-650-mm-1-mw}{laser} med bølgelengde på 650 nm
\end{itemize}
\vspace{0.2in}

\subsection{Metode, beregninger og observasjoner}
I et ideelt eksperimentelt oppsett ville man plassert et optisk gitter i en kjent avstand fra en vegg, strålet en laser gjennom gitteret og observert interferensmønsteret som ble dannet. Deretter kunne man målt avstanden mellom prikkene på veggen og bruke dette til videre beregninger, for eksempel til å finne spredningsvinklene. Det er i hvert fall slik jeg husker det fra da jeg gjorde dette forsøket i fysikken på videregående.

Siden jeg ikke får gjort forsøket, regner jeg med teoretiske verdier. 

\vspace{0.2in}

For å finne maksimumene på interferensmønsteret bruker vi formelen:
\begin{equation} 
    d \sin{\theta_m} = m \lambda \end{equation} \label{eq:32}
hvor:
\vspace{0.1in}
\begin{itemize}
    \item m er ordensnummeret fra maksimalpunktet. Jeg regner ut verdiene for $m = 1,2,3,4,5$. 
    \item d er spalteavstanden, gitt ved $d = \frac{1}{N}$
    \item N er linjetettheten til gitteret (i linjer per meter).
    \begin{itemize}
        \item $N_1 = 300 \times 10^3$ linjer/m
        \item $N_2 = 600 \times 10^3$ linjer/m
    \end{itemize}
    \item $\lambda $ = 650 nm som er bølgelengden til laseren.
    \item $\theta_m$ er spredningsvinkelen for hver orden av m.
    \item L er avstanden mellom spalten og veggen, som jeg setter til 1 meter. 
\end{itemize}


\begin{center} 
\includegraphics[width=0.3487\textwidth]{ny_ny_starten.png}
\includegraphics[width=0.3487\textwidth]{ny_ny_slutten.png}\\
\caption{\label{fig:starten på noe} Her viser jeg litt vagt hvordan man kommer frem til\\ formelen over når man har flere ordener av m.}
\end{center}



\begin{center}
\includegraphics[width=0.4\textwidth]{ny_ny_generell.png}\\
\caption{\label{fig:Generell} En litt mer generell og oversiktlig representasjon av noen ordner av m.}
\end{center}

\vspace{0.2in}

Når $L \gg d$ kan vi approksimere:
\begin{equation*}
    \tan{\theta} \approx \sin{\theta} \approx \theta \approx \frac{y_m}{L}
\end{equation*}

Her er $y_m$ avstanden fra sentrum til hvert interferensmaksimum på veggen, altså posisjonen til lysflekkene for hver orden $m$. 

\vspace{0.1in}
Fra tildligere vet vi at $d \sin{\theta} = m \lambda$. Da får vi:
\begin{align*}
    \sin{\theta} \approx \frac{y_m}{L} &= \frac{m \lambda}{d} \\
    y_m &= \frac{m \lambda L}{d} \\
    y_m &= \frac{m \lambda L}{\frac{1}{N}} \\
    y_m &= m \lambda L N \label{eq:33}\\
\end{align*}

\vspace{2in}

Med denne formelen kan vi regne ut verdier for $m = 1,2,3,4,5$. 

Beregner $y_m$ for $N_1 = 300 \cdot 10^3$ linjer/m
\begin{align*}
    y_1 = 1 \cdot 650 \cdot 10^{-9}\, m \cdot 1 \, m \cdot 300 \cdot 10^3 \, m^{-1} = \underline{\SI{0.195}{\meter}}\\
    y_2 = 2 \cdot 650 \cdot 10^{-9} \, m \cdot 1 \, m \cdot 300 \cdot 10^3 \, m^{-1} = \underline{\SI{0.390}{\meter}}\\
    y_3 = 3 \cdot 650 \cdot 10^{-9} \, m \cdot 1 \, m \cdot 300 \cdot 10^3 \, m^{-1} = \underline{\SI{0.585}{\meter}}\\
    y_4 = 4 \cdot 650 \cdot 10^{-9} \, m \cdot 1 \, m \cdot 300 \cdot 10^3 \, m^{-1} = \underline{\SI{0.780}{\meter}}\\
    y_5 = 5 \cdot 650 \cdot 10^{-9} \, m \cdot 1 \, m \cdot 300 \cdot 10^3 \, m^{-1} = \underline{\SI{0.975}{\meter} }
\end{align*}

\vspace{0.2in}

Regner nå videre $y_m$ for $N_2 = 600 \cdot 10^3$ linjer/m
\begin{align*}
    y_1 = 1 \cdot 650 \cdot 10^{-9} \, m \cdot 1 \, m \cdot 600 \cdot 10^3 \, m^{-1} = \underline{\SI{0.390}{\meter}}\\
    y_2 = 2 \cdot 650 \cdot 10^{-9} \, m \cdot 1 \, m \cdot 600 \cdot 10^3 \, m^{-1} = \underline{\SI{0.780}{\meter}}\\
    y_3 = 3 \cdot 650 \cdot 10^{-9} \, m \cdot 1 \, m \cdot 600 \cdot 10^3 \, m^{-1} = \underline{\SI{1.170}{\meter}}\\
    y_4 = 4 \cdot 650 \cdot 10^{-9} \, m \cdot 1 \, m \cdot 600 \cdot 10^3 \, m^{-1} = \underline{\SI{1.560}{\meter}}\\
    y_5 = 5 \cdot 650 \cdot 10^{-9} \, m \cdot 1 \, m \cdot 600 \cdot 10^3 \, m^{-1} = \underline{\SI{1.960}{\meter} }
\end{align*}

\vspace{0.2 in}
Nå som vi har verdier for $y_m$, kan vi bruke disse til å regne ut spredningsvinkelen $\theta_m$ for hver verdi og se om det blir noen signifikant endring. 

\vspace{0.1in}
Fra formelen $d \, \sin{\theta} = m \lambda$, når vi vet at $d = \frac{1}{N}$, kan vi løse for $\theta_m$ og få:
\begin{equation}
    \theta_m = \arcsin{(m \lambda N)}
\end{equation}

Beregner $\theta_m$ for $N_1 = 300 \cdot 10^3$ linjer/m
\[
\begin{array}{r l}
    \theta_1 &= \arcsin{(1 \cdot 650 \cdot 10^{-9} \, m \cdot 300 \cdot 10^3 \, m^{-1})} = \underline{11.25 ^\circ} \\
    \theta_2 &= \arcsin{(2 \cdot 650 \cdot 10^{-9} \, m \cdot 300 \cdot 10^3 \, m^{-1})} = \underline{22.95 ^\circ} \\
    \theta_3 &= \arcsin{(3 \cdot 650 \cdot 10^{-9} \, m \cdot 300 \cdot 10^3 \, m^{-1})} = \underline{35.80 ^\circ} \\
    \theta_4 &= \arcsin{(4 \cdot 650 \cdot 10^{-9} \, m \cdot 300 \cdot 10^3 \, m^{-1})} = \underline{51.26 ^\circ} \\
    \theta_5 &= \arcsin{(5 \cdot 650 \cdot 10^{-9} \, m \cdot 300 \cdot 10^3 \, m^{-1})} = \underline{77.16 ^\circ}
\end{array}
\]


Regner videre ut verdiene for $\theta_m$ for $N_2 = 600 \cdot 10^3$ linjer/m
\[
\begin{aligned}
    \theta_1 &= \arcsin{(1 \cdot 650 \cdot 10^{-9} \, m \cdot 600 \cdot 10^3 \, m^{-1})} = \underline{22.95 ^\circ} \\
    \theta_2 &= \arcsin{(2 \cdot 650 \cdot 10^{-9} \, m \cdot 600 \cdot 10^3 \, m^{-1})} = \underline{51.26 ^\circ} \\
    \theta_3 &= \arcsin{(3 \cdot 650 \cdot 10^{-9} \, m \cdot 600 \cdot 10^3 \, m^{-1})} \phantom{=} \text{ ikke definert for over } 90^\circ \\
    \theta_4 &= \arcsin{(4 \cdot 650 \cdot 10^{-9} \, m \cdot 600 \cdot 10^3 \, m^{-1})} \phantom{=} \text{ ikke definert for over } 90^\circ \\
    \theta_5 &= \arcsin{(5 \cdot 650 \cdot 10^{-9} \, m \cdot 600 \cdot 10^3 \, m^{-1})} \phantom{=} \text{ ikke definert for over } 90^\circ
\end{aligned}
\]

\vspace{0.2in}
Her er et forsøk på en illustrasjon av det vi har sett over. NB, vinklene er ikke riktige her.
\vspace{0.1in}
\begin{center}
\includegraphics[width=0.8\textwidth]{ny_ny_spaltene.png} \\
\caption{\label{fig:Spaltene} Til venstre er $N_1$ og til høyre er $N_2$. Vi ser at $N_2$ har større spredning i $y_m$.}
\end{center}

\vspace{0.2in}


\subsection{Drøfting og feilkilder}
Hadde vi utført forsøket fysisk, ville vi sett at interferensmønsteret for $N_2$ (600 linjer/mm) hadde større avstand mellom prikkene enn for $N_1$ (300 linjer/mm). Dette kan vi også bekrefte ut fra de teoretiske $y_m$-verdiene vi regnet ut, hvor $y_m$-verdiene til $N_2$ er mer spredt. Følgelig ser vi også at spredningsvinklene $\theta_m$ for $N_2$ er større enn for $N_1$, og ved høyere ordensnummer ($m > 2$) blir vinkelverdiene utilgjengelige fordi de overstiger 90$^\circ$. Dette betyr at spredningsvinkelen $\theta_m$ øker når spalteavstanden $d$ blir mindre, noe som kommer tydelig frem i beregningene. 

Siden jeg kun har utført teoretiske beregninger, har jeg valgt å ignorere praktiske feilkilder som skjelvende hender, en forsøkspartner med en uheldig tendens til å skumpe borti utstyret, eller feilmarginer fra linjaler. Jeg har også antatt at gittrene og laseren er ideelle og at de har nøyaktig de verdiene som de påstår å ha.


\vspace{0.15in}

\subsubsection{Sammenheng med Heisenbergs usikkerhetsprinsipp}
Heisenbergs usikkerhetsprinsipp, som jeg utledet tidligere (\ref{eq:31}), er gitt ved 
\begin{equation*}
    \Delta x \Delta p \ge \frac{\hbar}{2}
\end{equation*}

I eksperimentet tilsvarer spalteavstanden $d$ usikkerheten i posisjon $\Delta x$, mens spredningen i interferensmønsteret tilsvarer usikkerheten i bevegelsesmengde $\Delta p$. Når vi reduserer $d$ (eller øker linjetettheten $N$), observerer vi at spredningsvinkelen $\theta_m$ øker, noe som igjen fører til en større avstand mellom $y_m$-verdiene. Dette innebærer at fotonene får større usikkerhet i bevegelsesmengde, $\Delta p$. Dette er i samsvar med Heisenbergs usikkerhetsprinsipp: Jo mer presist vi bestemmer fotonets posisjon (ved å bruke en smalere spalte), desto større blir usikkerheten i dets bevegelsesmengde

Sammenlikning av de beregnede spredningsvinklene viser tydelig at for et gitter med høyere linjetetthet ($N_2$), blir interferensordenen med høyere $m$ raskt utilgjengelig fordi de overstiger 90$^\circ$. Dette illustrerer hvordan økt linjetetthet fører til større vinkelspredning, og dermed større usikkerhet i bevegelsesmengden $\Delta p$. På bildet \ref{fig:Spaltene} kan man også se at intensitetsgrafen 'flater ut' når linjetettheten øker. 


\subsection{Konklusjon}
Eksperimentet har nå vist at når man øker gitterkonstanten $N$, reduseres spalteavstanden $d$, noe som gjør at $\Delta x$ (usikkerheten i posisjon) blir mindre fordi vi kan måle posisjonen til fotonene mer presist. Som en konsekvens ser vi at interferensmønsteret sprer seg mer, noe som betyr at usikkerheten i bevegelsesmengden $\Delta p$ øker. Dette demonstrer Heisenbergs usikkerhetsprinsipp på en enkel måte, sett at du tilfeldigvis har en klasse II laser og et par opiske gittere liggende rundt :)

\vspace{0.2in}

\section{Siste ord} 
I denne rapporten har jeg grundig utforsket Heisenbergs usikkerhetsprinsipp, både gjennom matematisk utledning og teoretisk demonstrasjon ved hjelp av et enkelspalteforsøk. Gjennom dette har jeg fått en dypere forståelse av hvordan usikkerheten i posisjon og bevegelsesmengde er knyttet sammen. De teoretiske beregningene har vist hvordan endringer i spalteåpningen påvirker usikkerheten, noe som gir en konkret illustrasjon av prinsippet i praksis. 

Og som en liten bonus: mer erfaring med LaTeX! For hva er vel mer kvantemekanisk enn et system hvor små endringer i koden kan ha uforutsigbare makroskopiske konsekvenser?


\end{document}


